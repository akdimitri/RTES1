\documentclass[12pt, a4paper]{article}
\usepackage[LGR, T1]{fontenc}
\usepackage[utf8]{inputenc}
\usepackage[greek]{babel}
\usepackage{cite}
\usepackage{hyperref}
\usepackage{xcolor}
\usepackage[ruled, linesnumbered]{algorithm2e}
\usepackage[document]{ragged2e}
\usepackage{graphicx}
\graphicspath{ {./images/} }

\setlength{\parindent}{0em}

\title{{ΗΥ3604 Ενσωματωμένα Συστήματα Πραγματικού Χρόνου}\\
{\large Αριστοτέλειο Πανεπιστήμιο Θεσσαλονίκης}\newline \newline
{Εργασία 1\textsuperscript{η}}}
\author{Δημήτριος Αντωνιάδης\\8462
\\\textlatin{akdimitri@auth.gr}}

\date{Μάιος 2019}

\begin{document}

\maketitle

\newpage

\tableofcontents

\newpage

%1ο κεφάλαιο Εισαγωγή
%   1η Εισαγωγική Παράγραφος
\section{Εισαγωγή.}
\justify
\textbf{Τ}ο παρόν έγγραφο αποτελεί την ανφορά της πρωτης (1\textsuperscript{ης}) εργασίας που πραγματοποιήθηκε στο πλαίσιο του μαθήματος \textit{"Ενσωματωμένα Συστήματα Πραγματικού Χρόνου"}. Σκοπός της εργασίας ήταν να γίνει τακτική δειγματοληψία με την μικρότερη δυνατή απόκλιση από τον
πραγματικό χρόνο. Στο πείραμα αυτό οι τιμές της δειγματοληψίας ήταν τα \textlatin{\textbf{timestamps}} που επέστρεφε η συνάρτηση \textlatin{\textbf{gettimeofday()}}\cite{manual}.

%2η παράγραφος Δομή της εργασίας
\justify
Παραπάνω πραουσιάτηκε μία σύσντομη εισαγωγή αναφορικά με το θέμα της παρούσας αναφοράς. Στη δεύτερη (2\textsuperscript{η}) eνότητα παρουσιάζονται τα αρχεία που περιλαμβάνει η εργασία αυτή και η μέθοδος του \textit{\textlatin{Compilation}}. Στην τρίτη (3\textsuperscript{η}) ενότητα παρουσιάζονται οι δύο αλγόριθμοι που υλοποίηθηκαν και στην τέταρτη (4\textsuperscript{η}) ενότητα τα αποτελέσαμτα τους. Τέλος, στην πέμπτη (5\textsuperscript{η}) παρουσιάζονται τα συμπεράσματα της εργασίας αυτής.


%2ο Κεφάλαιο Περιεχόμενα Φακέλου Εργασίας και Μέθοδος Compilation
\section{Φάκελος υποβολής, \textlatin{Compilation} και Εκτέλεση.}
\textbf{Ο} φάκελος υποβολής με τον πηγαίο κώδικα βρίσκεται στην παρακάτω διεύθυνση:
%URL Github
\begin{itemize}
  \item \textlatin{\color{blue}{\url{https://github.com/akdimitri/RTES1}}}
\end{itemize}
Στον φάκελο \textit{\textlatin{code}} περιλαμβάνονται δύο (2) αρχεία. Τα αρχεία:
%Περιεχομενα ./code
\begin{itemize}
  \item \textlatin{simple.c}
  \item \textlatin{advanced.c}
\end{itemize}
\justify
Το πρώτο αρχείο αποτελεί το ζητούμενο πηγαίο κώδικα της εργασίας ο οποίος δεν κάνει χρήση των προηγούμενων
\textit{\textlatin{timestamps}}. Το δεύτερο αρχείο αποτελεί το ζητούμενο πηγαίο κώδικα της εργασίας ο οποίος κάνει χρήση των προηγούμενων \textit{\textlatin{timestamps}} με σκοπό την ακριβή δειγματοληψία πραγματικού χρόνου.
\justify
\textbf{Τ}ο \textit{\textlatin{Compilation}} των δύο παραπάνω αρχείων πραγματοποιείται με τις εξής εντολές:
%Compilation
\begin{itemize}
  \item \textlatin{gcc simple.c -o simple -O3}
  \item \textlatin{gcc advanced.c -o advanced -O3}
\end{itemize}

\justify
\textbf{Η} εκτέλεση και των δύο εκτελέσιμων αρχείων πραγματοποιείται με τον ίδιο τρόπο. Η εκτέλεση απαιτεί δύο ορίσματα. Το πρώτο όρισμα αντπροσωπεύει το χρόνο εκτέλεσης του πειράματος σε \textbf{ώρες} και το δεύτερο όρισμα το χρονικό διάστημα που μεσολαβει μεταξύ δύο δειγματοληψιών σε \textbf{δευτερόλεπτα}. Για παράδειγμα, με τις παρακάτω εντολές εκτελούνται τα πειράματα για 2 ώρες με χρονικό διάστημα μεταξύ δύο \textlatin{timestamps} ίσο με 0.1 δευτερόλεπτα:
\begin{itemize}
  \item \textlatin{./simple 2 0.1}
  \item \textlatin{./advanced 2 0.1}
\end{itemize}

\section{Αλγόριθμοι.}
\textbf{Σ}την παρούσα ενότητα παρουσιάζονται οι δύο αλγόριθμοι που πραγματοποιήθηκαν σε μορφή ψευδογλώσσας ώστε να είναι εύκολα κατανοητοί.

\subsection{Αλγόριθμος \textlatin{simple}}
\textbf{Σ}την υποενότητα αυτή παρουσιάζεται ο αλγόριθμος που υλοποιείται από τον πηγαίο κώδικα \textlatin{\textit{simple.c}}. Το πρόγραμμα δέχεται ως όρισμα το συνολικό χρόνο εκτέλεσης του προγράμματος (\textlatin{\textit{executionTime}}) και το διάστημα (\textlatin{\textit{interval}}) που μεσολαβεί μεταξυ δύο δειγματοληψιών χρόνου. Συνεπώς, είναι δυνατό διαιρώντας το συνολικό χρόνο εκτέλεσης με το χρονικό διάστημα διαμεσολάβησης να βρεθεί ο συνολικός αριθός χρονικων διαστηματων διαμεσολάβησης μεταξύ των δειγματοληψιών. Για να πραγματοποιηθούν τόσα διαστήματα διαμεσολάβησης απαιτείται να ληφθούν ισόποσα στιγμιότυπα χρόνου (\textlatin{\textit{timestamps}} συν ένα (1) επιπλέον. Η μεταβλητή που αναπαριστά τον αριθμό των συνολικων στιγμιοτύπων που θα ληφθούν ονομάζεται \textlatin{\textit{iterations}}.
\justify
\textbf{Η} δειγματοληψία των χρονικών στιγμών πραγματοποιείται με τη χρήση της συνάρτησης
\begin{itemize}
\item \textlatin{\textit{gettimeofday(struct timeval *tv,
    struct timezone *tz)}}\cite{manual}
\end{itemize}
η οποία επιστρέφει σε κατάλληλη δομή (\textlatin{struct}) το χρόνο που έχει παρέλθει από τη χρονική στιγμή \textlatin{\textit{Epoch}}, δηλαδή τη χρονική στιγμή 1970-01-01 00:00:00 +0000 (\textlatin{UTC}).
\justify
\textbf{Τ}ο διάστημα που μεσολαβεί μεταξύ της δειγματοληψίας δύο χρονικών στιγμών πραγματοποιείται με τη χρήση της συνάρτησης
\begin{itemize}
    \item \textlatin{\textit{usleep( useconds\_t usec)}}\cite{manual}
\end{itemize}
Η συνάρτηση αυτή αναστέλλει την εκτέλεση του \textlatin{thread} που εκτελείται για χρονικό διάστημα \textlatin{\textit{usec}}.
\newline
\justify
\textlatin{
\begin{algorithm}[H]
\SetKwData{Left}{left}\SetKwData{This}{this}\SetKwData{Up}{up}
\SetKwFunction{Union}{Union}\SetKwFunction{FindCompress}{FindCompress}
\SetKwInOut{Input}{input}\SetKwInOut{Output}{output}
\Input{$executionTime(HOURS), interval(SECONDS)$}
\Output{$timestamps$ MATRIX[$iterations$, 1] }
 $iterations \longleftarrow (executionTime)*3600/interval + 1$
 \BlankLine
 \BlankLine
 $timestamps[1] \longleftarrow gettimeofday(...)$\\
 \For{$i \leftarrow 2$ : $iterations$}{
  usleep($interval*1000000$)\\
  $timestamps[i] \longleftarrow gettimeofday(...)$
 }
 export $timestamps$ MATRIX to a text file
 \caption{simple.c}
\end{algorithm}}


\subsection{Αλγόριθμος \textlatin{advanced}.}
\textbf{Ο} αλγόριθμος αυτός δειγματοληπτεί τόσες χρονικές στιγμές όσες και ο παραπάνω. Η μόνη διαφορά είναι οτι ο παραπάνω αλγόριθμος αποτυγχάνει τα δειγματοληπτήσει με την ακρίβεια που δειγματοληπτεί ο advanced. Ο αλγόριθμος που παρετίθεται παρακάτω κάνει χρήση των προηγούμενων χρονικών στιγμών που έχουν ληφθεί κατα τη δειγματοληψία ώστε να βελτιώσει την απόδοση του. Συγκεκριμένα, εφόσον είναι γνωστή η πρώτη τιμή δειγματοληψίας και το χρονικό διάστημα που μεσολαβεί μεταξύ δύο δειγματοληψιών είναι δυνατό να γνωρίζουμε και την ακριβή στιγμή που πρέπει να πραγματοποιηθεί η δειγματοληψία της τιμής $i$. Δηλαδή,\newline 

\centering{$timestamps[i] \longleftarrow timestamps[1] + i*interval$}

\justifyΕπομένως, για κάθε τιμή δειγματοληψίας είναι γνωστή και η αναμενόμενη τιμή της. Έτσι, για κάθε τιμή δειγματοληψίας μπορεί να υπολογιστεί και η χρονική απόκλιση και να αφαιρεθεί από το επόμενο χρονικό διάστημα μεσολάβησης.
\newline

\justify
\textlatin{
\begin{algorithm}[H]
\SetKwData{Left}{left}\SetKwData{This}{this}\SetKwData{Up}{up}
\SetKwFunction{Union}{Union}\SetKwFunction{FindCompress}{FindCompress}
\SetKwInOut{Input}{input}\SetKwInOut{Output}{output}
\Input{$executionTime(HOURS), interval(SECONDS)$}
\Output{$timestamps$ MATRIX[$iterations$, 1] }
 $iterations \longleftarrow (executionTime)*3600/interval + 1$\\
 $delay \longleftarrow 0$
 \BlankLine
 \BlankLine
 $timestamps[1] \longleftarrow gettimeofday(...)$\\
 \For{$i \leftarrow 2$ : $iterations$}{
  usleep($interval*1000000 - delay$)\\
  $timestamps[i] \longleftarrow gettimeofday(...)$
  $delay \longleftarrow timestamps[i] - i*interval - timestamps[0]$
 }
 export $timestamps$ MATRIX to a text file
 \caption{simple.c}
\end{algorithm}}
\newpage
%4η Ενοτητα Αποτελέσματα Πειραματων.
\section{Αποτελέσματα.}
\justify
\textbf{Σ}την ενότητα αυτή παρουσιάζονται τα αποτελέσματα από την εκτέλεση των παραπάνω αλγορίθμων. Τα παρακάτω διαγράμματα προκύπτουν ύστερα από την επεξεργασία των δεδομένων μέσω του προγράμματος \textlatin{\textbf{\textit{Rstudio}}}. Το \textlatin{\textit{script.r}} περιλαμβάνει όλες τις εντολές για την εξαγωγή των παρακάτω αποτελεσμάτων και διαγραμματών.

\subsection{Αποτελέσματα αλγορίθμου \textlatin{simple}.}
\justify
\textbf{H} εκτέλεση του αλγορίθμου αυτού με ορίσματα:
\begin{itemize}
    \item $executionTime \longleftarrow 2$
    \item $interval \longleftarrow 0.1$
\end{itemize}
είχε συνολικό χρόνο: $7207.682$ $secs$ δηλαδή καθυστέρησε για περίπου \textbf{7.5} δευτερόλεπτα. Η μέση τιμή ήταν \textbf{100106.7 \textlatin{usec}}, δηλαδή μεταξύ κάθε κάθε μέτρησης υπήρχε καθυστέρηση περίπου 107 μικροδευτερολέπτων. Η τυπική απόκλιση ήταν \textbf{18.024}.

\begin{figure}[h]
\caption{Τιμές δειγματοληψίας αλγορίθμου \textlatin{simple}}
\centering
\includegraphics[width=0.5\textwidth]{images/simple.png}
\end{figure}
\begin{figure}[h]
\caption{Κατανομή τιμών αλγορίθμου \textlatin{simple}}
\centering
\includegraphics[width=0.5\textwidth]{images/simple_distribution.png}
\end{figure}

\newpage

\justify
\textbf{Ο}πως φαίνεται και από τα παραπάνω διαγράμματα στο σύνολό τους οι τιμές ήταν μεγαλύτερες των 100000 \textlatin{usec} που ήταν η επιθυμητή τιμή, για το λόγο αυτό παρατηρήθηκε και αυτή η απόκλιση. 

\newpage
\subsection{Αποτελέσματα αλγορίθμου \textlatin{advanced}.}
\justify
\textbf{H} εκτέλεση του αλγορίθμου αυτού με ορίσματα:
\begin{itemize}
    \item $executionTime \longleftarrow 2$
    \item $interval \longleftarrow 0.1$
\end{itemize}
είχε συνολικό χρόνο: $7200.000108$ $secs$ δηλαδή καθυστέρησε για περίπου \textbf{0.000108} δευτερόλεπτα ή 108 μικροδευτερόλεπτα. Η μέση τιμή ήταν \textbf{100000.0015 \textlatin{usec}}, δηλαδή μεταξύ κάθε κάθε μέτρησης υπήρχε καθυστέρηση περίπου 0.0015 μικροδευτερολέπτων. Η τυπική απόκλιση ήταν \textbf{9.14}.


\begin{figure}[h]
\caption{Τιμές δειγματοληψίας αλγορίθμου \textlatin{advanced}}
\centering
\includegraphics[width=0.5\textwidth]{images/advanced.png}
\end{figure}
\begin{figure}[h]
\caption{Κατανομή τιμών αλγορίθμου \textlatin{advanced}}
\centering
\includegraphics[width=0.5\textwidth]{images/advanced_distribution.png}
\justify
\textbf{Ο}πως φαίνεται και από τα παραπάνω διαγράμματα στο σύνολό του οι τιμές κυμάνθηκαν περίπου στα 100000 μικροδευτερόλεπτα που ήταν και η επιθυμητή τιμή.
\end{figure}

\newpage


\section{Συμπεράσματα}
Τελικά, η εργασία αυτή καταλήγει στο συμπέρασμα οτι η δειγματοληψία τιμών την οποία εκτελεί ένα ενσωματωμένο σύστημα είναι καλό να ελέγχεται και να γίνεται προσπάθεια να είναι όσο το δυνατό ακριβέστερη. Στην παρούσα υλοποίηση παρατηρήθηκε σφάλμα της τάξης του 
\bibliographystyle{plain}
\bibliography{M335}

\end{document}
